\documentclass[a4paper, 10pt, fleqn]{article}

\usepackage[utf8]{inputenc}
\usepackage[T1]{fontenc}
\usepackage{textcomp}
\usepackage{lmodern}
\usepackage[ngerman]{babel}
\usepackage{enumerate}
\usepackage[landscape,top=10pt,bottom=50pt,left=10pt,right=10pt]{geometry} %for horizontal aligned documents
%\usepackage[top=100pt,bottom=100pt,left=100pt,right=100pt]{geometry}

%table packages
\usepackage{tabularx}
\usepackage{array}
\usepackage{rotating}
\usepackage{multicol}

%mathe packages
\usepackage{amsmath}

%figure packages
\usepackage{subcaption}
\usepackage{caption}
\usepackage{float}
\usepackage{graphicx}

%bibliography packages
\usepackage{hyperref}
\usepackage{apacite}

\usepackage{color}

%for displaying source code
\usepackage{listings}
\lstset{
                basicstyle=\footnotesize,
                keywordstyle=\color{blue},
                stringstyle=\color{red},
                commentstyle=\color{green},
}

\usepackage{float}
%====================================
%============OWN COMMANDS==============
%====================================

%============PAGE PROPERTIES=============
\newcommand{\revisiondate}{\today}
\newcommand{\documenttitle}{document title}
\newcommand{\authors}{Me}
\newcommand{\subtitle}{subtitle}


%==============BOXES==================
%small colored box with no visible border, breaks before and after box
\definecolor{shadecolor}{RGB}{200,200,200}
\definecolor{lightgreen}{RGB}{0,50,0}
\newcommand{\shadebox}[1]{\par\noindent\colorbox{shadecolor}
{\parbox{\dimexpr\textwidth-2\fboxsep\relax}{#1}}}
%small noncolored box with visible border, breaks before and after box
\newcommand{\commandbox}[1]{\par\noindent\fbox{\begin{minipage}{\textwidth}#1\end{minipage}}\par\noindent}
%small colored box with breaks and border
\newcommand{\lessons}[1]{\par\noindent\colorbox{lightgreen}\fbox{\begin{minipage}{\textwidth}#1\end{minipage}}\par\noindent}
%====================================

%==========TITLE================
\newcommand*{\titleGM}{\begingroup % Create the command for including the title page in the document
\hbox{ % Horizontal box
\hspace*{0.2\textwidth} % Whitespace to the left of the title page
\rule{1pt}{\textheight} % Vertical line
\hspace*{0.05\textwidth} % Whitespace between the vertical line and title page text
\parbox[b]{0.75\textwidth}{ % Paragraph box which restricts text to less than the width of the page

{\noindent\Huge\bfseries \documenttitle}\\[2\baselineskip] % Title
{\large \textit{\subtitle}}\\[4\baselineskip] % Tagline or further description
{\Large \textsc{\authors}} % Author name

\vspace{0.5\textheight} % Whitespace between the title block and the publisher
{\noindent \revisiondate}\\[\baselineskip] 
}}
\endgroup}
%=============================

%==========HEADER & FOOTER=======
%=============================

%===========TABLE COMMANDS=======
\renewcommand\tabularxcolumn[1]{>{\small}m{#1}}
\newcolumntype{Y}{>{\centering\arraybackslash}X}
\newcommand\RotText[1]{\rotatebox{90}{\parbox{2cm}{\centering#1}}}
\renewcommand{\arraystretch}{1.5}

\newlength{\arrayrulewidthOriginal}
\newcommand{\Cline}[2]{%
  \noalign{\global\setlength{\arrayrulewidthOriginal}{\arrayrulewidth}}%
  \noalign{\global\setlength{\arrayrulewidth}{#1}}\cline{#2}%
  \noalign{\global\setlength{\arrayrulewidth}{\arrayrulewidthOriginal}}}
%=============================


\begin{document}

\begin{table}
	\centering
	\begin{tabular}{cp{6cm}p{6cm}p{6cm}p{6cm}}
	\textbf{Level} & \textbf{Entsagung} & \textbf{Beschwörung} &
	\textbf{Prophezeiung} & \textbf{Verzauberung}\\ \hline	
	2 & magische barriere (Ward) wenn ein entsagungszauber gewirkt wird (Ward's HP = level + int) erneutes wirken eines zaubers heilt den ward um 2*zauberlevel HP. Einmal pro lange Rast
	& Beschwören eines nicht-lebendigen Objekt innerhalb von 10ft, Hält 1h an ausser es wird attackiert. Leuchtet leicht im Dunkeln und darf maximal 5kg wiegen und 1m breit sein.
	& Nach einer langer Rast würfle 2 D20. Einmal pro zug kann (bevor gewürfelt wird) das Ergebnis eines Würfelwurfs mit einem der beiden d20's ausgetauscht werden.
	& Verzaubere ein Wesen innerhalb von 5ft (Weisheit Rettungswurf - Zauberer's Wirkungs DC) sodass es sich nicht bewegen kann. Dauer 1-2 rundnen oder ausserhalb der Reichweite.\\ \hline
	6 & Der Ward kann nun Schaden von Lebewesen innerhalb von 30ft absorbieren. Überzähliger Schaden wird weiter aufgeteilt.
	& Kann innerhalb von 30ft teleportieren (Auch mit anderen Plätze tauschen). Einmal pro lange Rast oder Beschwörungszauber lvl 1+ gespielt wurde.
	& Prophezeiungszauber benötigen 1 slot weniger als normal (mind. 1 slot).
	& Kann einen Angriff weiterleiten (Weisheits Rettungswurf - Zauberer's Wirkungs DC)\\ \hline
	10 & + prof Boni bei Ability checks von Entsagungszaubern.
	& Deine Konzentration kann nicht durch Schaden unterbrochen werden
	& Als Aktion deine Wahrnehmung verstärken. Erhalte einen von folgenden Boni bis zu einer kurzen oder langen Rast
	\begin{itemize}
		\item Darkvision: Du siehst im Dunkeln
		\item Ethereal Sight: Du siehst in die Etherische Ebene
		\item Greater Comprehension: Du kannst alle Sprachen lesen
		\item See Invisibility: Du siehst unsichtbare Kreaturen innerhalb von 10ft
	\end{itemize}
	& Verzauberungszauber welche 1 Kreatur als Ziel haben, können nun 2 Ziele haben.\\ \hline
	14 & Vorteil bei Rettungswürfen gegen Zauber
	& Jede Beschwörung (Kreatur) hat zusätzliche 30 HP
	& Du kannst nun 3 D20 anstelle 2 werfen
	& 1 verzaubertes Wesen merkt nicht, dass es verzaubert ist und Du kannst sein Gedächtnis auslöschen (maximal zu 1h, Intelligenz Rettungswurf - Zauberer's Wirkungs DC) 
	\end{tabular}
	\caption{die verschiednen Zaubererschulen, part 1}
\end{table}
\clearpage
\begin{table}
	\centering
	\begin{tabular}{cp{6cm}p{6cm}p{6cm}p{6cm}}
	\textbf{Level} & \textbf{Evokation} & \textbf{Illusion} &
	\textbf{Nekromagie} & \textbf{Transmuation}\\ \hline
	
	2 & Du kannst 1+Zaubererlevel an sicheren Stellen im Wirkungsbereich des Zaubers schaffen. Wesen darin bestehen Rettungswürfe automatisch und nehmen keinen Schaden.
	& Du erlernst den Zauber \textit{Minor Illusion} (falls du den bereits kennst, füge einen anderen Cantrip hinzu). Dieser cantrip zählt nicht zur Maximalanzahl.
	Beim ausführen von \textit{Minor Illusion} kannst du gleichzeitig ein Geräusch und ein Bild erzeugen
	& Einmal pro Zug erhälst du HP gleich des Zauberlevels * 2 der zum töten einer Kreatur verwendet wurde. 3*Zauberlevel, wenn der Zauber Nekromantie war
	& Du kannst ein Objekt aus Holz, Stein, Eisen, Kupfer oder Silber bestehend umwandeln, dass es aus einem anderen der oben genannten Materialien ist.
	Für jede 10 Minuten Ausführung, kann 3 Kubikdecimeter Material verwandelt werden.
	\\ \hline
	
	6 & Deine Cantrips teilen auch bei gelungenen Rettungswürfen Schaden aus (die Hälfte als Normal) die Wesen erleiden aber keinen weiteren Effekt.
	& Du kannst die Natur von Illusionszauber verändern.
	& Füge \textit{animate dead} deinem Zauberbuch hinzu. Du darfst einen zusätzlichen Körper oder Knochenhaufen als Ziel nehmen. Zombie oder Skelette erhalten zusätzlich
	\begin{itemize}
		\item HP gleich des Zaubererlevels
		\item deinen Fertigkeitsbonus als austeilender Schaden
	\end{itemize}
	& Während 8h erstellst du einen Transmutations Stein. Beim Erstellen wählst du einen der folgenden Effekte für den Träger:
	\begin{itemize}
		\item Darkvision auf bis zu 60ft
		\item Erhöhung der Geschwindigkeit um 10 Fuss
		\item Fertigkeit in Konstitutionsrettungswürfen
		\item Resistenz auf Säure, Kälte, Feuer, Blitz oder Donner
	\end{itemize}
	Der Effekt kann beim Wirken eines Transmutationszauber Level 1 oder höher geändert werden. Das Erstellen eines neuen Steins, zerstört den alten.
	\\ \hline
	
	10 & Füge deinen Intelligenz Modifier zum Schaden hinzu, den ein Evokationszauber zufügt
	& Du kannst als Reaktion eine Illusion von dier erstellen, die den Schaden abfängt und sich danach auflöst. Einmal pro lange Rast
	& Du bist nun resistent auf nekrotischen Schaden und dein HP maximum kann nicht reduziert werden.
	& Füge \textit{Polymorph} deinem Zauberbuch hinzu. Einmal pro Rast kannst du dich selbst in ein Tier verwandeln (Challenge rating von 1 oder tiefer). Verbraucht keinen slot wenn er auf sich selbst angewandt wird.
	\\ \hline
	
	14 & Jeder Schadenszauber unter Level 6, fügt den maximal Schaden aus. Einmal pro Lange Rast ohne Eigenschaden einsetzbar. Für jedes weitere ausführen erleidest du 2d12 nekrotischen Schaden
	& Du kannst Teile (Nicht-magische, nicht-lebende Teile) von Illusionen für 1 Minute real werden lassen. Diese können keinen Schaden austeilen.
	& Du kannst einen Untoten innerhalb von 60ft befehligen (Charisma Rettungswurf - Zauberer's Wirkungs DC). Intelligente (8+) haben Vorteil beim Rettungswurf, 12+ können den Wurf jede 1h wiederholen
	& Du kannst alle Energie im Transmutations Stein aufbrauchen, ihn zerstören und einen der folgenden Effekte erhalten
	\begin{itemize}
		\item Verwandle ein 1.5m grosses Objekt
		\item Entferne jeden Fluch, Krankheit und Vergiftung und heile komplett das Ziel
		\item Stelle Jugend oder Leben her
	\end{itemize}
	\\
	\end{tabular}
	\caption{die verschiednen Zaubererschulen, part 2}
\end{table}

\clearpage
\begin{table}
	\centering
	\begin{tabular}{cp{6cm}p{6cm}}
	\textbf{Level} & \textbf{College of Valor} & \textbf{College of Lore}\\ \hline
	3 &
	Geübt in Shilder, Mittlere Rüstung und Martial Waffen \linebreak
	Eine Kreatur mit Bardischem Inspirationswürfel darf den Würfel für das Verbessern der Armor Class (AC) oder
	zum erhöhen eines Schadens brauchen. &
	Geübt in 3 wählbaren Fähigkeiten.\linebreak
	Die Bardische Inspiration kann benutzt werden um einen Angriffswurf, Ability Check oder Schadenswurf zu reduzieren. Das Ziel is immun wenn es nicht hören kann oder nicht bezaubert werden kann.\\ \hline
	6 &
	Du kannst 2mal angreifen anstatt 1mal &
	Du lernst 2 beliebige Zauber welche deinem Zauberlevel entsprechen. (Zählen nicht zur maximalen Zauberanzahl)\\ \hline
	14 &
	Du kannst als Bonusaktion mit deiner Waffe angreifen, wenn du als Aktion einen Zauber gespielt hast. &
	Du kannst deine Bardische Inspiration auf einen Ability Check anwenden um das Ergebnis zu verbessern.
	\end{tabular}
	\caption{Die 2 Bardschulen}
\end{table}
\clearpage
\begin{table}
	\centering
	\begin{tabular}{cp{6cm}p{6cm}}
	\textbf{Level} & \textbf{Landzirkel} & \textbf{Mondzirkel} \\ \hline
	& \textit{zusätzliche Zaubersprüche bei Lvl 3, 5, 7 und 9, siehe 2. tabelle} & \\ \hline
	2 &
	1 zusätzlicher Zaubertrick (Cantrip) deiner Wahl \linebreak
	Du kannst während einer kurzen Rast kombinierte Anzahl an Zauberslot auffrischen = der Hälfte des Druidenlevels und maximal ein 6. Levelslot. &
	Du kannst ''Wildshape'' als Bonusaktion ausführen. Zusätzlich kannst du einen Zauberslot aufbrauchen um 1d8 * Zauberlevel an Lebenspunkten wiederherzustellen. Wildshape hat nun ein maximales Challenge Rating = Druidlevel / 3 (abgerundet, min. 1). \\ \hline
	
	6 &
	Du kannst ungehindert durch nicht-magisches erschwertes Terrain laufen. Du kannst zwischen Nicht-magischen Pflanzen durchlaufen, ohne Schaden zu nehmen.\linebreak
	Du hast Vorteil bei Rettungswürfen auf magische Pflanzen welche Bewegung einschränken. &
	Angriffe in der Beastform zählen als magisch\\ \hline
	
	10 &
	Immun vor Gift und Krankheiten. Du kannst nicht mehr von Feen oder Elementaren eingeschüchtert oder bezaubert werden &
	Du kannst 2 Verwendungen einsetzen um dich in ein Wasser, Feuer, Luft oder Erdelementar zu verwandeln. \\ \hline
	
	14 &
	Kreaturen oder Pflanzen, die dich angreifen, müssen einen Weisheitsrettungswurf machen (DC = Druiden Zauber DC). Bei Misslingen, müssen sie ein anderes Ziel nehmen oder verfehlen automatisch. Bei Gelingen ist das Wesen für 24h immun gegen diesen Effekt. &
	Du kannst den Zauber ''alter self'' nach freiem Wilen einsetzen.\\
	\end{tabular}
	\caption{die verschiedenen Druidenzirkel}
\end{table}

\clearpage
\begin{table}
	\centering
	\begin{tabular}{cp{3cm}p{3cm}p{3cm}p{3cm}p{3cm}p{3cm}p{3cm}p{3cm}}
	\textbf{Level} & \textbf{Arktik} & \textbf{Küste} & \textbf{Wüste} & \textbf{Wald} & \textbf{Grasland} & \textbf{Gebirge} & \textbf{Sumpf} & \textbf{Underdark} \\ \hline
	3 & 
	\begin{itemize}
		\item Hold Person
		\item Spike Growth
	\end{itemize}&
	\begin{itemize}
		\item Mirror Image
		\item misty Step 
	\end{itemize}& 
	\begin{itemize}
		\item Blur
		\item Silence
	\end{itemize}&
	\begin{itemize}
		\item Barkskin
		\item Spider climb
	\end{itemize}&
	\begin{itemize}
		\item Invisibility
		\item pass w/o trace
	\end{itemize}&
	\begin{itemize}
		\item Spider climb
		\item spike growth
	\end{itemize}&
	\begin{itemize}
		\item darkness
		\item Melf's acid arrow
	\end{itemize}&
	\begin{itemize}
		\item Spider climb
		\item web
	\end{itemize}\\ \hline
	
	5 & 
	\begin{itemize}
		\item sleet storm
		\item slow
	\end{itemize}&
	\begin{itemize}
		\item water breathing
		\item water walk
	\end{itemize}& 
	\begin{itemize}
		\item create food and water
		\item protection from energy
	\end{itemize}&
	\begin{itemize}
		\item call lightning
		\item plant growth
	\end{itemize}&
	\begin{itemize}
		\item daylight
		\item haste
	\end{itemize}&
	\begin{itemize}
		\item lightning bold
		\item meld into stone
	\end{itemize}&
	\begin{itemize}
		\item water walk
		\item stinking cloud
	\end{itemize}&
	\begin{itemize}
		\item gaseous form
		\item stinking cloud
	\end{itemize}\\ \hline

	7 & 
	\begin{itemize}
		\item freedom of movement
		\item ice storm
	\end{itemize}&
	\begin{itemize}
		\item control water
		\item freedom of movement
	\end{itemize}& 
	\begin{itemize}
		\item blight
		\item hallucinatory terrain
	\end{itemize}&
	\begin{itemize}
		\item divination
		\item freedom of movement
	\end{itemize}&
	\begin{itemize}
		\item divination
		\item freedom of movement
	\end{itemize}&
	\begin{itemize}
		\item stone shape
		\item stoneskin
	\end{itemize}&
	\begin{itemize}
		\item freedom of movement
		\item locate creature
	\end{itemize}&
	\begin{itemize}
		\item greater invisibility
		\item stone shape
	\end{itemize}\\ \hline
	
	9 & 
	\begin{itemize}
		\item commune with nature
		\item cone of cold
	\end{itemize}&
	\begin{itemize}
		\item conjure elemental
		\item scrying
	\end{itemize}& 
	\begin{itemize}
		\item insect plague
		\item wall of stone
	\end{itemize}&
	\begin{itemize}
		\item commune with nature
		\item tree stride
	\end{itemize}&
	\begin{itemize}
		\item dream
		\item insect plague
	\end{itemize}&
	\begin{itemize}
		\item passwall
		\item wall of stone
	\end{itemize}&
	\begin{itemize}
		\item insect plague
		\item scrying
	\end{itemize}&
	\begin{itemize}
		\item cloudkill
		\item insect plague
	\end{itemize}\\ \hline		
	\end{tabular}
	\caption{Zaubersprüche für den Landzirkel}
\end{table}

\end{document}