\begin{table}
	\centering
	\begin{tabular}{cp{6cm}p{6cm}p{6cm}}
	\textbf{Level} & \textbf{Champion} & \textbf{Battle Master} & \textbf{Eldritch Knight}\\ \hline

	3 &
	Bei Angriffswürfen zählen 19 und 20 als kritische Treffer&
	Du bist geübt in einem Handwerkszeug deiner Wahl\linebreak
	Du erhälst 4 ''Superiority Würfel'' (D8) welche du einsetzen kannst um einmal pro Zug ein Maneuver auszuüben. Wähle 3 Maneuver aus (Siehe Liste der Maneuver) \linebreak
	\textbf{Maneuver DC =  8 + prof. bonus + stärke oder geschicklichkeit}&
	Du lernst 2 Zaubertrick (Zauberer) und 3 Zaubersprüche (Wizard, 2 davon aus abjuration und evocation)\linebreak
	Du kannst als Ritual einen Bund mit einer beliebigen Waffe eingehen. Diese kannst du als Bonusaktion heraufbeschwören und du kannst nicht entwaffnet werden.\\ \hline
	
	7 &
	Du kannst den halben Fertigkeitsboni zu Konstitution, Geschicklichkeit und Stärke checks dazuzählen, wenn du nicht bereits geübt bist darin\linebreak
	Deine Sprungdistanz ist um deine Stärke modifizierer in Fuss erweitert&
	Du erlernst zwei weitere Maneuver (bereits bekannte können getauscht werden) und ein superior Würfel zusätzlich\linebreak
	Du studierst eine Minute lang eine Kreatur ausserhalb eines Kampfes und erfährst wie weit sie dir überlegen ist&
	Wenn du ein Zaubertrick anwendest, kannst du als Bonusaktion einen Waffenangriff starten\\ \hline
	
	10 &
	Du kannst einen weiteren Kampfstil auswählen&
	Deine Superior würfel werden zu D10s &
	Ein Waffenangriff verursacht bei einer Kreatur Nachteil beim Rettungswurf auf einen von dir ausgeführten Zauber\\ \hline
	
	15 &
	Bei Angriffswürfen zählen 18-20 als kritische Treffer&
	Wenn du Initiative würfelst und keine Superior Würfel mehr hast, erhälst du einen&
	\\ \hline
	
	18 &
	Wenn du wenigers als die Hälfte an Leben hast, erhälst du zu Beginn deines Zuges Lebenspunkte = 5 + Konstitutionsmodifikator&
	Deine Superior Würfel werden zu D12&
	\\ \hline
	\end{tabular}
	\caption{Die verschiedenen Martial Archetypes}
\end{table}
