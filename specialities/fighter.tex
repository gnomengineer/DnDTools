\begin{table}
	\centering
	\begin{tabular}{cp{6cm}p{6cm}p{6cm}}
	\textbf{Level} & \textbf{Champion} & \textbf{Battle Master} & \textbf{Eldritch Knight}\\ \hline

	3 &
	Bei Angriffswürfen zählen 19 und 20 als kritische Treffer&
	Du bist geübt in einem Handwerkszeug deiner Wahl\linebreak
	Du erhälst 4 ''Superiority Würfel'' (D8) welche du einsetzen kannst um einmal pro Zug ein Maneuver auszuüben. Wähle 3 Maneuver aus (Siehe Liste der Maneuver) \linebreak
	\textbf{Maneuver DC =  8 + prof. bonus + stärke oder geschicklichkeit}&
	\\ \hline
	
	7 &
	Du kannst den halben Fertigkeitsboni zu Konstitution, Geschicklichkeit und Stärke checks dazuzählen, wenn du nicht bereits geübt bist darin\linebreak
	Deine Sprungdistanz ist um deine Stärke modifizierer in Fuss erweitert&
	Du erlernst ein weiteres Maneuver (bereits bekannte können getauscht werden)\linebreak
	Du &
	\\ \hline
	
	10 &
	Du kannst einen weiteren Kampfstil auswählen&
	&
	\\ \hline
	
	15 &
	Bei Angriffswürfen zählen 18-20 als kritische Treffer&
	&
	\\ \hline
	
	18 &
	Wenn du wenigers als die Hälfte an Leben hast, erhälst du zu Beginn deines Zuges Lebenspunkte = 5 + Konstitutionsmodifikator&
	&
	\\ \hline
	\end{tabular}
	\caption{Die verschiedenen Martial Archetypes}
\end{table}

\begin{tabulary}{19cm}{LLLL}
	test & test & Du bist geübt in einem Handwerkszeug deiner Wahl\linebreak
	Du erhälst 4 ''Superiority Würfel'' (D8) welche du einsetzen kannst um einmal pro Zug ein Maneuver auszuüben. Wähle 3 Maneuver aus (Siehe Liste der Maneuver)
	\textbf{Maneuver DC =  8 + prof. bonus + stärke oder geschicklichkeit} \\
\end{tabulary}