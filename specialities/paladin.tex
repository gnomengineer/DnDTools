\begin{table}
	\centering
	\begin{tabular}{cp{6cm}p{6cm}p{6cm}}
	\textbf{Level} & \textbf{Schwur der Hingabe} & \textbf{Schwur der Alten} & \textbf{Schwur der Rache}\\ \hline

	3 &
	Die Lehre der Ehrlichkeit, Mut, Mitgefühl, Ehre und Pflicht\linebreak\linebreak
	\textbf{Geistliche Waffe}: Benutze X um eine Waffe zu verzaubern. Füge den Charismamodifier dem Angriffswurf hinzu und die Waffe erstrahlt in einem Licht\linebreak
	\textbf{Wende Unheiliges}: Benutze X. Jeder Untote und Unhold muss einen Weisheitsrettungswurf machen. Bei Misserfolg läuft er bei jedem seiner Züge von dir weg und bleibt min. 30 Fuss von dir entfernt. Es kann nur ausweichen oder sprint aktionen ausführen&
	Die Lehre von Licht entzünden, beschützen, bewahren und vorleben\linebreak\linebreak
	\textbf{Naturs Zorn}: Benutze X um einen Gegner innerhalb von 10 Fuss in Ranken zu fangen. Der Gegner muss einen Stärke oder Beweglichkeitsrettungswurf machen (wiederholung am Ende seines Zuges)\linebreak
	\textbf{Bekehren der Ungläubigen}: Benutze X. Jeder Untote und Unhold muss einen Weisheitsrettungswurf machen. Bei Misserfolg läuft er bei jedem seiner Züge von dir weg und bleibt min. 30 Fuss von dir entfernt. Es kann nur ausweichen oder sprint aktionen ausführen. Illusionen zeigen ihre wahre Form&
	Die Lehre vom Kampf gegen das Grosse Böse, keine Gnade fürs boshafte, Entschädigung, um jeden Preis\linebreak\linebreak
	\textbf{Dem Gegner entsagen}: Als Aktion kannst ud X benutzen um einen Gegner innerhalb von 60 Fuss zu erschrecken. Bei Erfolg eines Weisheitsrettungswurf (Untote und Unholde haben Nachteil) wird die Geschwindigkeit halbiert, sonst auf 0 reduziert\linebreak\textbf{Gelübde der Verfeindung}: Du kannst als Bonusaktion ein Gelübde sprechen (verwendet X) und somit Vorteil bei Angriffswürfen erhält\\ \hline

	7 &
	Du und freundliche Wesen innerhalb eines 10 Fuss radius können nicht bezaubert werden (bei Level 18 sind es 30 Fuss)&
	Du und freundliche Wesen innerhalb eines 10 Fuss radius haben Resistenz gegen Zauber (bei Level 18 sind es 30 Fuss)&
	Wenn du mit einem Gelegenheitsangriff triffst, darfst du dich bewegen = Halbe Anzahl an Geschwindigkeit (Löst keinen Gelegenheitsangriff aus)\\ \hline
	
	15 &
	Du hast immer den Effekt des \textit{protection from evil and good} Zauber auf dir&
	Falls du auf 0 Lebenspunkte reduziert wirst, kannst du stattdessen auf 1 Lebenspunkt reduziert werden (einmal pro lange Rast). Du erhälst keine Beeinträchtigung durch dein Alter&
 	Wenn Gegner unter dem Gelübde der Verfeindung angreifen, darfst du als Reaktion einen Nahkampfangriff gegen ihn ausführen\\ \hline
	
	20 &
	1mal pro lange Rast. Du strahlst im Sonnenlicht für 1 minute 30 Fuss weit. Feindliche Kreaturen welche den Zug im Licht beginnen, nehmen 10 radiant Schaden. Du hast Vorteil auf Rettungswürfe gegen Zauber von Untoten &
	Als Aktion kannst du die Form einer antike Naturform annehmen. Du erhälst für 1 Minute folgende Eigenschaften
	\begin{itemize}
		\item Du erhälst bei Zuganfang 10 Lebenspunkte
		\item Paladinzauber, die maximal eine Aktion brauchen, können als Bonusaktion verwenden
		\item Gegner näher als 10 Fuss haben Nachteile bei Paladinzaubern und X
	\end{itemize}&
	Du kannst als Aktion dich in eine engelhaften Rächer verwandeln und erhälst für 1 Minute folgende Effekte
	\begin{itemize}
		\item Du kannst nun 60 Fuss hoch fliegen
		\item Du strahlst eine einschüchternde Aura aus (30 Fuss). Kreaturen darin müssen einen Weisheittrettungswurf machen, bei Misserfolg sind sie eingeschüchtert und du hast Vorteil auf Angriffswürfe gegen sie.
	\end{itemize}		
	\\ \hline	
	\end{tabular}
	\caption{die verschiedenen Schwüre eines Paladins}
\end{table}