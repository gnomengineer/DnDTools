\begin{table}
	\centering
	\begin{tabular}{cp{6cm}p{6cm}p{6cm}p{6cm}}
	\textbf{Level} & \textbf{Entsagung} & \textbf{Beschwörung} &
	\textbf{Prophezeiung} & \textbf{Verzauberung}\\ \hline	
	2 & magische barriere (Ward) wenn ein entsagungszauber gewirkt wird (Ward's HP = level + int) erneutes wirken eines zaubers heilt den ward um 2*zauberlevel HP. Einmal pro lange Rast
	& Beschwören eines nicht-lebendigen Objekt innerhalb von 10ft, Hält 1h an ausser es wird attackiert. Leuchtet leicht im Dunkeln und darf maximal 5kg wiegen und 1m breit sein.
	& Nach einer langer Rast würfle 2 D20. Einmal pro zug kann (bevor gewürfelt wird) das Ergebnis eines Würfelwurfs mit einem der beiden d20's ausgetauscht werden.
	& Verzaubere ein Wesen innerhalb von 5ft (Weisheit Rettungswurf - Zauberer's Wirkungs DC) sodass es sich nicht bewegen kann. Dauer 1-2 rundnen oder ausserhalb der Reichweite.\\ \hline
	6 & Der Ward kann nun Schaden von Lebewesen innerhalb von 30ft absorbieren. Überzähliger Schaden wird weiter aufgeteilt.
	& Kann innerhalb von 30ft teleportieren (Auch mit anderen Plätze tauschen). Einmal pro lange Rast oder Beschwörungszauber lvl 1+ gespielt wurde.
	& Prophezeiungszauber benötigen 1 slot weniger als normal (mind. 1 slot).
	& Kann einen Angriff weiterleiten (Weisheits Rettungswurf - Zauberer's Wirkungs DC)\\ \hline
	10 & + prof Boni bei Ability checks von Entsagungszaubern.
	& Deine Konzentration kann nicht durch Schaden unterbrochen werden
	& Als Aktion deine Wahrnehmung verstärken. Erhalte einen von folgenden Boni bis zu einer kurzen oder langen Rast
	\begin{itemize}
		\item Darkvision: Du siehst im Dunkeln
		\item Ethereal Sight: Du siehst in die Etherische Ebene
		\item Greater Comprehension: Du kannst alle Sprachen lesen
		\item See Invisibility: Du siehst unsichtbare Kreaturen innerhalb von 10ft
	\end{itemize}
	& Verzauberungszauber welche 1 Kreatur als Ziel haben, können nun 2 Ziele haben.\\ \hline
	14 & Vorteil bei Rettungswürfen gegen Zauber
	& Jede Beschwörung (Kreatur) hat zusätzliche 30 HP
	& Du kannst nun 3 D20 anstelle 2 werfen
	& 1 verzaubertes Wesen merkt nicht, dass es verzaubert ist und Du kannst sein Gedächtnis auslöschen (maximal zu 1h, Intelligenz Rettungswurf - Zauberer's Wirkungs DC) 
	\end{tabular}
	\caption{die verschiednen Zaubererschulen, part 1}
\end{table}
\clearpage
\begin{table}
	\centering
	\begin{tabular}{cp{6cm}p{6cm}p{6cm}p{6cm}}
	\textbf{Level} & \textbf{Evokation} & \textbf{Illusion} &
	\textbf{Nekromagie} & \textbf{Transmuation}\\ \hline
	
	2 & Du kannst 1+Zaubererlevel an sicheren Stellen im Wirkungsbereich des Zaubers schaffen. Wesen darin bestehen Rettungswürfe automatisch und nehmen keinen Schaden.
	& Du erlernst den Zauber \textit{Minor Illusion} (falls du den bereits kennst, füge einen anderen Cantrip hinzu). Dieser cantrip zählt nicht zur Maximalanzahl.
	Beim ausführen von \textit{Minor Illusion} kannst du gleichzeitig ein Geräusch und ein Bild erzeugen
	& Einmal pro Zug erhälst du HP gleich des Zauberlevels * 2 der zum töten einer Kreatur verwendet wurde. 3*Zauberlevel, wenn der Zauber Nekromantie war
	& Du kannst ein Objekt aus Holz, Stein, Eisen, Kupfer oder Silber bestehend umwandeln, dass es aus einem anderen der oben genannten Materialien ist.
	Für jede 10 Minuten Ausführung, kann 3 Kubikdecimeter Material verwandelt werden.
	\\ \hline
	
	6 & Deine Cantrips teilen auch bei gelungenen Rettungswürfen Schaden aus (die Hälfte als Normal) die Wesen erleiden aber keinen weiteren Effekt.
	& Du kannst die Natur von Illusionszauber verändern.
	& Füge \textit{animate dead} deinem Zauberbuch hinzu. Du darfst einen zusätzlichen Körper oder Knochenhaufen als Ziel nehmen. Zombie oder Skelette erhalten zusätzlich
	\begin{itemize}
		\item HP gleich des Zaubererlevels
		\item deinen Fertigkeitsbonus als austeilender Schaden
	\end{itemize}
	& Während 8h erstellst du einen Transmutations Stein. Beim Erstellen wählst du einen der folgenden Effekte für den Träger:
	\begin{itemize}
		\item Darkvision auf bis zu 60ft
		\item Erhöhung der Geschwindigkeit um 10 Fuss
		\item Fertigkeit in Konstitutionsrettungswürfen
		\item Resistenz auf Säure, Kälte, Feuer, Blitz oder Donner
	\end{itemize}
	Der Effekt kann beim Wirken eines Transmutationszauber Level 1 oder höher geändert werden. Das Erstellen eines neuen Steins, zerstört den alten.
	\\ \hline
	
	10 & Füge deinen Intelligenz Modifier zum Schaden hinzu, den ein Evokationszauber zufügt
	& Du kannst als Reaktion eine Illusion von dier erstellen, die den Schaden abfängt und sich danach auflöst. Einmal pro lange Rast
	& Du bist nun resistent auf nekrotischen Schaden und dein HP maximum kann nicht reduziert werden.
	& Füge \textit{Polymorph} deinem Zauberbuch hinzu. Einmal pro Rast kannst du dich selbst in ein Tier verwandeln (Challenge rating von 1 oder tiefer). Verbraucht keinen slot wenn er auf sich selbst angewandt wird.
	\\ \hline
	
	14 & Jeder Schadenszauber unter Level 6, fügt den maximal Schaden aus. Einmal pro Lange Rast ohne Eigenschaden einsetzbar. Für jedes weitere ausführen erleidest du 2d12 nekrotischen Schaden
	& Du kannst Teile (Nicht-magische, nicht-lebende Teile) von Illusionen für 1 Minute real werden lassen. Diese können keinen Schaden austeilen.
	& Du kannst einen Untoten innerhalb von 60ft befehligen (Charisma Rettungswurf - Zauberer's Wirkungs DC). Intelligente (8+) haben Vorteil beim Rettungswurf, 12+ können den Wurf jede 1h wiederholen
	& Du kannst alle Energie im Transmutations Stein aufbrauchen, ihn zerstören und einen der folgenden Effekte erhalten
	\begin{itemize}
		\item Verwandle ein 1.5m grosses Objekt
		\item Entferne jeden Fluch, Krankheit und Vergiftung und heile komplett das Ziel
		\item Stelle Jugend oder Leben her
	\end{itemize}
	\\
	\end{tabular}
	\caption{die verschiednen Zaubererschulen, part 2}
\end{table}
