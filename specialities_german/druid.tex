\begin{table}
	\centering
	\begin{tabular}{cp{6cm}p{6cm}}
	\textbf{Level} & \textbf{Landzirkel} & \textbf{Mondzirkel} \\ \hline
	& \textit{zusätzliche Zaubersprüche bei Lvl 3, 5, 7 und 9, siehe 2. tabelle} & \\ \hline
	2 &
	1 zusätzlicher Zaubertrick (Cantrip) deiner Wahl \linebreak
	Du kannst während einer kurzen Rast kombinierte Anzahl an Zauberslot auffrischen = der Hälfte des Druidenlevels und maximal ein 6. Levelslot. &
	Du kannst ''Wildshape'' als Bonusaktion ausführen. Zusätzlich kannst du einen Zauberslot aufbrauchen um 1d8 * Zauberlevel an Lebenspunkten wiederherzustellen. Wildshape hat nun ein maximales Challenge Rating = Druidlevel / 3 (abgerundet, min. 1). \\ \hline
	
	6 &
	Du kannst ungehindert durch nicht-magisches erschwertes Terrain laufen. Du kannst zwischen Nicht-magischen Pflanzen durchlaufen, ohne Schaden zu nehmen.\linebreak
	Du hast Vorteil bei Rettungswürfen auf magische Pflanzen welche Bewegung einschränken. &
	Angriffe in der Beastform zählen als magisch\\ \hline
	
	10 &
	Immun vor Gift und Krankheiten. Du kannst nicht mehr von Feen oder Elementaren eingeschüchtert oder bezaubert werden &
	Du kannst 2 Verwendungen einsetzen um dich in ein Wasser, Feuer, Luft oder Erdelementar zu verwandeln. \\ \hline
	
	14 &
	Kreaturen oder Pflanzen, die dich angreifen, müssen einen Weisheitsrettungswurf machen (DC = Druiden Zauber DC). Bei Misslingen, müssen sie ein anderes Ziel nehmen oder verfehlen automatisch. Bei Gelingen ist das Wesen für 24h immun gegen diesen Effekt. &
	Du kannst den Zauber ''alter self'' nach freiem Wilen einsetzen.\\
	\end{tabular}
	\caption{die verschiedenen Druidenzirkel}
\end{table}

\clearpage
\begin{table}
	\centering
	\begin{tabular}{cp{3cm}p{3cm}p{3cm}p{3cm}p{3cm}p{3cm}p{3cm}p{3cm}}
	\textbf{Level} & \textbf{Arktik} & \textbf{Küste} & \textbf{Wüste} & \textbf{Wald} & \textbf{Grasland} & \textbf{Gebirge} & \textbf{Sumpf} & \textbf{Underdark} \\ \hline
	3 & 
	\begin{itemize}
		\item Hold Person
		\item Spike Growth
	\end{itemize}&
	\begin{itemize}
		\item Mirror Image
		\item misty Step 
	\end{itemize}& 
	\begin{itemize}
		\item Blur
		\item Silence
	\end{itemize}&
	\begin{itemize}
		\item Barkskin
		\item Spider climb
	\end{itemize}&
	\begin{itemize}
		\item Invisibility
		\item pass w/o trace
	\end{itemize}&
	\begin{itemize}
		\item Spider climb
		\item spike growth
	\end{itemize}&
	\begin{itemize}
		\item darkness
		\item Melf's acid arrow
	\end{itemize}&
	\begin{itemize}
		\item Spider climb
		\item web
	\end{itemize}\\ \hline
	
	5 & 
	\begin{itemize}
		\item sleet storm
		\item slow
	\end{itemize}&
	\begin{itemize}
		\item water breathing
		\item water walk
	\end{itemize}& 
	\begin{itemize}
		\item create food and water
		\item protection from energy
	\end{itemize}&
	\begin{itemize}
		\item call lightning
		\item plant growth
	\end{itemize}&
	\begin{itemize}
		\item daylight
		\item haste
	\end{itemize}&
	\begin{itemize}
		\item lightning bold
		\item meld into stone
	\end{itemize}&
	\begin{itemize}
		\item water walk
		\item stinking cloud
	\end{itemize}&
	\begin{itemize}
		\item gaseous form
		\item stinking cloud
	\end{itemize}\\ \hline

	7 & 
	\begin{itemize}
		\item freedom of movement
		\item ice storm
	\end{itemize}&
	\begin{itemize}
		\item control water
		\item freedom of movement
	\end{itemize}& 
	\begin{itemize}
		\item blight
		\item hallucinatory terrain
	\end{itemize}&
	\begin{itemize}
		\item divination
		\item freedom of movement
	\end{itemize}&
	\begin{itemize}
		\item divination
		\item freedom of movement
	\end{itemize}&
	\begin{itemize}
		\item stone shape
		\item stoneskin
	\end{itemize}&
	\begin{itemize}
		\item freedom of movement
		\item locate creature
	\end{itemize}&
	\begin{itemize}
		\item greater invisibility
		\item stone shape
	\end{itemize}\\ \hline
	
	9 & 
	\begin{itemize}
		\item commune with nature
		\item cone of cold
	\end{itemize}&
	\begin{itemize}
		\item conjure elemental
		\item scrying
	\end{itemize}& 
	\begin{itemize}
		\item insect plague
		\item wall of stone
	\end{itemize}&
	\begin{itemize}
		\item commune with nature
		\item tree stride
	\end{itemize}&
	\begin{itemize}
		\item dream
		\item insect plague
	\end{itemize}&
	\begin{itemize}
		\item passwall
		\item wall of stone
	\end{itemize}&
	\begin{itemize}
		\item insect plague
		\item scrying
	\end{itemize}&
	\begin{itemize}
		\item cloudkill
		\item insect plague
	\end{itemize}\\ \hline		
	\end{tabular}
	\caption{Zaubersprüche für den Landzirkel}
\end{table}